\chapter{Solvers}
\section{Introduction}
The solver for the ODE problem can be specified using the 'Solver' option of the \sourceword{PPODE\_build} function:
\matlabcmd{PPODE\_build(\ph{source}, \ph{target}, 'Solver', \ph{solver})}
Valid options for \ph{solver} are:
\begin{description}
 \item['Stiff' (or 'BDF')] The BDF based solver of the LSODE package. \emph{See \ref{subsec:LSODEBDF}}.
 \item['Stiff2' (or 'CVODE')] The BDF based solver of the CVODE package. \emph{See \ref{subsec:CVODEBDF}}.
 \item['MEBDFSO' (or 'MEBDFSparse')] The modified extended BDF based solver using a sparse Jacobian matrix. \emph{See \ref{subsec:MEBDFSO}}.
 \item['LSODES' (or 'BDFSparse')] The BDF based solver using a sparse Jacobian matrix. \emph{See \ref{subsec:LSODES}}.
 \item['Non-Stiff' (or 'Adams-Moulton')] The Adams-Moulton based solver of the LSODE package. \emph{See \ref{subsec:LSODEAM}}.
 \item['Non-Stiff2' (or 'CVODEAM')] The Adams-Moulton based solver of the CVODE package. \emph{See \ref{subsec:CVODEAM}}.
 \item['RK23', 'RK45', 'RK78'] The Runge-Kutta based solvers of the RKSUITE package. \emph{See \ref{subsec:RKSUITE}}.
 \item['Switching' (or 'LSODA')] The solver that switches between the non-stiff Adams-Moulton based solver and the stiff BDF based solver of the LSODE package. \emph{See \ref{subsec:LSODA}}.
\end{description}

If you do not know which solver to choose, but you already know which \MATLAB solver performs best, table \ref{tab:MATLABToPPODE} might be helpfull.
\begin{table}
\begin{tabular}{ | l || l | p{5cm} | } \hline
 \textbf{\MATLAB} & \textbf{Equivalent} & \textbf{Probably Also Suitable} \\
 \hline
 \hline
 ode45 & 'RK45' & 'RK78', 'Non-Stiff', 'Non-Stiff2' \\ \hline
 ode23 & 'RK23' & 'RK45', 'Non-Stiff', 'Non-Stiff2' \\ \hline
 ode113 & 'Non-Stiff', 'Non-Stiff2' & 'RK45' \\
 \hline
 \hline
 ode15s & 'Stiff', 'Stiff2' & 'Switching' (partially stiff problems), 'MEBDFSO' (large number of states that are not very interdependent) \\ \hline
 ode23s & - & 'Stiff', 'Stiff2', 'Switching', 'MEBDFSO' \\ \hline
 ode23t & - & 'Switching', 'Stiff', 'Stiff2', 'MEBDFSO' \\ \hline
 ode23tb & - & 'Switching', 'Stiff', 'Stiff2', 'RK45', 'RK23', 'MEBDFSO' \\ \hline
\end{tabular} 
\caption{Solver selection helper.\label{tab:MATLABToPPODE}}
\end{table}

\section{Stiff}
\subsection{BDF}
\label{subsec:LSODEBDF}
The Backward Differential Formulas based method uses the BDF implementation that ODEPACK supplies. The order of these formulae can range between 1 and 5 and can be limited by setting the 'MaxOrder' option when building the ODE system.

\subsubsection{Credits}
Credits for the ODEPACK package obtained from (\hyperlink{http://www.netlib.org/}).

\vspace{0.5cm}
\begin{tabular}{ l l }
 \textbf{Author}      & Alan C. Hindmarsh \\
 \textbf{Institution} & Center for Applied Scientific Computing, L-561 \\
                      & Lawrence Livermore National Laboratory \\
                      & Livermore, CA 94551 \\
                      & United States of America \\
\end{tabular}
\subsection{Modified Extended BDF using Sparse Jacobian}
\label{subsec:MEBDFSO}
The Modified Extended Backward Differential Formulae based method uses the BDF implementation that MEBDFSO supplies. The order of these formulas can range between 1 and 5 and can be limited by setting the 'MaxOrder' option when building the ODE system.

\subsubsection{Credits}
Credits for the MEBDFSO package obtained from (\hyperlink{http://www.netlib.org/}).

\vspace{0.5cm}
\begin{tabular}{ l l }
 \textbf{Authors}     & T.J. Abdulla \\
                      & J.R. Cash \\
 \textbf{Institution} & Department of Mathematics \\
                      & Imperial College \\
                      & London SW7 2AZ \\
                      & England \\
 \textbf{Contact}     & t.abdulla@ic.ac.uk \\
                      & j.cash@ic.ac.uk \\
\end{tabular}

\section{Non-Stiff}
\subsection{Adams-Moulton Methods}
\label{subsec:LSODEAM}
The Adams-Moulton method based solver uses the Adams-Moulton implementation that ODEPACK supplies. The order of these formulae can range between 1 and 12 and can be limited by setting the 'MaxOrder' option when building the ODE system.
\subsubsection{Credits}
Credits for the ODEPACK package obtained from (\hyperlink{http://www.netlib.org/}).

\vspace{0.5cm}
\begin{tabular}{ l l }
 \textbf{Author}      & Alan C. Hindmarsh \\
 \textbf{Institution} & Center for Applied Scientific Computing, L-561 \\
                      & Lawrence Livermore National Laboratory \\
                      & Livermore, CA 94551 \\
                      & United States of America \\
\end{tabular}

\subsection{Runge-Kutta Methods}
\label{subsec:RKSUITE}
The Runga-Kutta methods based solver uses the RKSUITE package. Three Runge-Kutta pairs are available: 2-3, 4-5 and 7-8. Use higher orders in combination with smaller tolerances.
\subsubsection{Credits}
Credits for the RKSUITE package obtained from (\hyperlink{http://www.netlib.org/}).

\vspace{0.5cm}
\begin{tabular}{ l l }
 \textbf{Author}      & R.W. Brankin \\
 \textbf{Institution} & Numerical Algorithms Group Ltd. \\
                      & Wilkinson House \\
                      & Jordan Hill Road \\
                      & Oxford OX2 8DR \\
                      & United Kingdom \\
 \textbf{Contact}     & richard@nag.co.uk \\
                      & na.brankin@na-net.ornl.gov \\
 \textbf{Authors}     & I. Gladwell \\
                      & L.F. Shampine \\
 \textbf{Institution} & Department of Mathematics \\
                      & Southern Methodist University \\
                      & Dallas, Texas 75275 \\
                      & United States of America \\
 \textbf{Contact}     & h5nr1001@vm.cis.smu.edu \\
\end{tabular}
\section{Mixed}
\subsection{Switching between BDF and Adams-Moulton Methods}
\label{subsec:LSODA}
The switching method uses the LSODA subroutine that the ODEPACK package supplies. This method automatically switches between the BDF based stiff solver (order 1-5) and the Adams-Moulton methods based non-stiff solver (order 1-12).

\subsubsection{Credits}
Credits for the ODEPACK package obtained from (\hyperlink{http://www.netlib.org/}) and the LSODA subroutine in particular.

\vspace{0.5cm}
\begin{tabular}{ l l }
 \textbf{Author}      & Alan C. Hindmarsh \\
 \textbf{Institution} & Center for Applied Scientific Computing, L-561 \\
                      & Lawrence Livermore National Laboratory \\
                      & Livermore, CA 94551 \\
                      & United States of America \\
 \textbf{Author}      & Linda R. Petzold \\
 \textbf{Institution} & Univ. of California at Santa Barbara \\
                      & Dept. of Computer Science \\
                      & Santa Barbara, CA 93106 \\
                      & United States of America \\
\end{tabular}
