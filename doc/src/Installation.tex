\chapter[Installation]{Installation}
The \PPODESUITE package depends on:
\begin{description}
  \item[\MATLAB] Tested on \MATLAB version R2013b (8.2) 64-bit.
  \item[gfortran (GCC)] Tested on version 4.8.1 64-bit.
\end{description}
In order to fulfill these requirements under \MSDOS, the ... package is needed.

\section[\nix]{Linux and other Unix variants}
\begin{enumerate}
 \item Meet the software requirements by installing \MATLAB and the GCC package. \MATLAB download and installation instructions can be found on the \MATHWORKS website. The gfortran/GCC package can be obtained from the \nix distribution repository through the distribution package manager. Using \UBUNTU for example:
 \shellcmd{sudo apt-get install gfortran}
\end{enumerate}

\section[Windows]{\MSDOS}
\begin{enumerate}
 \item Do windows stuff
 \begin{enumerate}
  \item Nested item 1
  \item Nested item 2
 \end{enumerate}
\end{enumerate}

\section{General}
\begin{enumerate}[resume]
 \item Download and extract the \PPODESUITE package.
 \item Open matlab and navigate to the extracted \PPODESUITE folder. Add the \PPODE paths to the matlab path variable.
 \matlabcmd{PPODE\_addPaths}
 \item Now the libraries of the different solvers can be build. In order to do so, execute the following command;
 \matlabcmd{PPODE\_init}
 The options 'Debug' can be used to build the libraries with debugging symbols.
 \matlabcmd{PPODE\_init('Debug', 1)}
\end{enumerate}
